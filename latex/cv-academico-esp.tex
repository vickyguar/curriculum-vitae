\documentclass[a4paper,10pt]{article}
\usepackage{config/config-esp} % Carga toda tu configuración profesional

\begin{document}
\pagestyle{empty} 

% HEADER
\begin{center}
    {\Huge \scshape{\color{primaryColor}Victoria Guarnieri}} \\ [5pt]
    {\Large {Ingeniera Biomédica}} \\ [8pt]
    \small
    \href{mailto:vickyguarnieri1@gmail.com}{\faEnvelope \ vickyguarnieri1@gmail.com} \ $|$ \ 
    \href{tel:+5491130410475}{\faPhone \ +54 9 11 3041-0475} \ $|$ \ 
    \faMapMarker* \ Buenos Aires, Argentina \\ [3pt]
    \href{https://github.com/vickyguar}{\faGithub \ vickyguar} \ $|$ \ 
    \href{https://linkedin.com/in/victoria-guarnieri-}{\faLinkedin \ victoria-guarnieri-} \ $|$ \ 
    \faIdCard \ DNI 43.660.111
\end{center}

% FORMACIÓN ACADÉMICA
\section{Formación Académica}
\begin{formacion}{Ingeniería Biomédica}{Facultad de Ingeniería y Ciencias Exactas y Naturales}{Universidad Favaloro (FICEN-UF)}{Mar 2020 -- Dic 2025}
    \item \textbf{Promedio:} 8,35/10.
    \item \textbf{Proyecto Final:} Gestión de Datos en Investigación Oncológica: Desarrollo de Software Integral para Ensayos \textit{In Vivo}.
\end{formacion}
\tecnologias{Python, C/C++, C\#, MATLAB, Microsoft SQL Server.}\\
\habilidades{Trabajo en equipo, investigación, resolución de problemas, disciplina, perseverancia.}

% EXPERIENCIA LABORAL
\section{Experiencia Laboral}
\begin{experiencia}{Programa de Inteligencia Artificial en Salud, Hospital Italiano de Buenos Aires}
    \begin{rol}{Trainee en IA}{Sep 2024 -- Presente}
        \item Construcción y evaluación de un modelo de aprendizaje profundo para clasificación.
        \item Diseño y ejecución de un estudio de nivel de acuerdo interobservador para definir etiquetas de entrenamiento.
        \item Documentación técnica y seguimiento de experimentos.
        \item Evaluación de productos externos.
    \end{rol}
\end{experiencia}
% \tecnologiasItems{
%     \item \textbf{Python 3.13:} PyTorch, Pandas, Numpy, Scikit-learn, Matplotlib, Seaborn. \textbf{R 4.3:} tidyverse, ggplot2, plotly, irrCAC.
%     \item \textbf{Cloud:} AWS SageMaker y AWS S3.
%     \item \textbf{Control de versiones:} Git, Github y Bitbucket.
%     \item \textbf{Seguimiento de experimentos:} Weights \& Biases (wandb).
% }
% \habilidades{Análisis y visualización de datos, aprendizaje automático, aprendizaje profundo, procesamiento de imágenes, estadística aplicada.}
\tecnologias{\textbf{Python 3.13} (PyTorch, Pandas, Numpy, Scikit-learn, Matplotlib, Seaborn, Pyplot); \textbf{AWS services} (AWS SageMaker, S3); \textbf{control de versiones} con Git, Github, Bitbucket; \textbf{seguimiento de experimentos} con Weights \& Biases (wandb); \textbf{R 4.3} (tidyverse, ggplot2, plotly, irrCAC).}\\
\habilidades{Análisis y visualización de datos, aprendizaje automático, aprendizaje profundo, procesamiento de imágenes, estadística aplicada.}

\vspace{5pt}
\begin{experiencia}{FICEN-UF -- Departamento de Matemática}
    % \begin{rol}{Jefa de trabajos prácticos (JTP)}{Mar 2026 -- Presente}
    %     \item JTP en la materia \textbf{Cálculo III} (Cálculo en variable compleja).
    %     \item Dictado de clases teórico-prácticas.
    %     \item Desarrollo de materiales de estudio y guías de ejercicios.
    %     \item Escritura, evaluación y corrección de exámenes parciales.
    % \end{rol}
    \begin{rol_corto}{Ayudante Alumno}{Ago 2021 -- Mar 2026}
        Ayudante en las materias \textbf{Cálculo I} (Cálculo en una variable), \textbf{Cálculo II} (Cálculo en varias variables) y \textbf{Cálculo III} (Cálculo en variable compleja).
    \end{rol_corto}
\end{experiencia}

\vspace{5pt}
\begin{experiencia}{FICEN-UF -- Departamento de Informática}
    \begin{rol}{Ayudante Alumno}{Mar 2025 -- Jul 2025}
        \item Ayudante en la materia \textbf{Base de Datos}.
        \item Colaboración en la escritura, evaluación y corrección de exámenes parciales y trabajos prácticos, bajo supervisión.
    \end{rol}
\end{experiencia}
\tecnologias{MySQL 8.0.} \\
\habilidades{Modelado y normalización de bases de datos, programación en SQL.}

\vspace{5pt}
\begin{experiencia}{Hospital Universitario Fundación Favaloro}
    \begin{rol}{Rotante en Ingeniería Biomédica}{May 2024 -- Sep 2024}
        \item Práctica profesional supervisada.
        \item Tareas de servicio técnico preventivo y correctivo. Reparación a nivel placa y a nivel componentes; control de funcionamiento con testers biomédicos; control de seguridad eléctrica con analizador; compra de componentes a través del fondo fijo del sector.
        \item Actualización de protocolos, conforme a la norma IRAM 15 y análisis de datos históricos de mantenimiento.
    \end{rol}
\end{experiencia}
\tecnologias{Microsoft Excel, Neovero.} \\
\habilidades{Ingeniería Clínica.}

% BECAS Y RECONOCIMIENTOS
\section{Becas y Reconocimientos}
\begin{itemize}[leftmargin=1.2em, label=\small\faAward, nosep]
    \item \normalsize\textbf{Beca de Investigación para Estudiantes (BIE)}, UF \fecha{Feb 2022 -- Feb 2025}
    \item \normalsize\textbf{Segunda escolta de bandera nacional}, FICEN-UF \fecha{Periodo lectivo 2024}
\end{itemize}

\pagebreak

% CURSOS Y MINICURSOS
\section{Cursos}
\curso{Deep Learning: Redes neuronales desde cero}{Centro de e-Learning UTN FRBA. Con evaluación.}{37}{Jul 2023}
\curso{Introducción al aprendizaje automático no supervisado}{FICEN-UF}{16}{May 2023}
\curso{Machine Learning con Python}{Centro de e-Learning UTN FRBA. Con evaluación.}{60}{May 2023}
\curso{Introducción a Python orientado a algoritmos y técnicas de programación competitiva}{IEEE UF, IEEE WIE y LambdaClass}{20}{Nov 2020}
% \curso{Modelos Aditivos Generalizados (GAM)}{Reunión del Grupo Argentino de Bioestadística}{4}{Octubre de 2025}
% \curso{Por qué (¿no?) me gustaría/quisiera/debería ``convertirme'' a la Estadística bayesiana y nunca me atreví a preguntármelo… }{Reunión del Grupo Argentino de Bioestadística}{3}{Octubre de 2025}
% \curso{Ideas estocásticas fundamentales que conectan el análisis exploratorio y la inferencia informal}{Reunión del Grupo Argentino de Bioestadística}{3}{Octubre de 2025}

% ASISTENCIA A CONGRESOS
\section{Asistencia a Congresos y Jornadas}
\evento{XX Jornadas de Informática en Salud JIS Summit del Hospital Italiano de Buenos Aires}{Centro Metropolitano de Diseño, Ciudad Autónoma de Buenos Aires, Argentina. Evento híbrido}{7 al 9 de noviembre de 2025}
\evento{Jornada de actualización sobre Patología Vulvar de la Sociedad Iberoamericana de Vulva y Vagina (SIAVV)}{Yacht Club Puerto Madero, Ciudad Autónoma de Buenos Aires, Argentina}{24 de octubre de 2025}
\evento{XXV Congreso Argentino de Bioingeniería, las XIV Jornadas de Ingeniería Clínica y la III Conferencia Latinoamericana de Ingeniería Clínica, SABI-2025}{Hotel 13 de Julio, Mar del Plata, Argentina}{14 al 17 de octubre de 2025}
\evento{XXIX Reunión Científica del Grupo Argentino de Bioestadística, GAB 2025}{Universidad Nacional del Nordeste, Corrientes, Argentina}{1 al 3 de octubre de 2025}
\evento{II Jornadas de enseñanza de la Estadística}{Universidad Nacional del Nordeste, Corrientes, Argentina}{30 de septiembre de 2025}
\evento{XVIII Jornadas de Informática en Salud JIS Summit del Hospital Italiano de Buenos Aires}{Centro Metropolitano de Diseño, Ciudad Autónoma de Buenos Aires, Argentina. Evento híbrido}{1 al 3 de noviembre de 2023}
\evento{XXIV Congreso Argentino de Bioingeniería y las XIII Jornadas de Ingeniería Clínica, SABI 2023}{Universidad Nacional de Arturo Jauretche, Florencio Varela, y Centro Cultural Kirchner, Ciudad Autónoma de Buenos Aires, Argentina}{3 al 6 de octubre de 2023}
\evento{XXIII Congreso Argentino de Bioingeniería y las XII Jornadas de Ingeniería Clínica, SABI 2022}{Universidad Nacional de San Juan, San Juan, Argentina}{13 al 16 de septiembre de 2022}

\section{Presentaciones de Pósteres}
\evento{\normalsize{Determinación de grado de acuerdo para variables categóricas: a propósito de un caso}}{GAB 2025}{Rusconi Lagarrigue A.B., Sguiglia S., \textbf{Guarnieri V.}, et al}
\evento{\normalsize{Determinación del grado de acuerdo interobservador en segmentación de imágenes médicas: a propósito de un caso}}{GAB 2025}{Sguiglia S., Rusconi Lagarrigue A.B., \textbf{Guarnieri V.}, et al}

\section{Presentaciones Orales}
\evento{\normalsize{Estudio de acuerdo interobservador en la valoración de imágenes de lesiones por presión}}{JIS SUMMIT 2025}{Rusconi Lagarrigue A.B., Sguiglia S., \textbf{Guarnieri V}}
\evento{\normalsize{Inteligencia artificial en la práctica médica}}{Jornada SIAVV}{Caridi J., \textbf{Guarnieri V}}

% IDIOMAS Y SKILLS
\section{Idioma}
\textbf{Inglés} (Competencia básica, lectura técnica y escritura científica).

% VOLUNTARIADO 
\section{Voluntariado}
\begin{experiencia}{Capítulo estudiantil de la Sociedad Argentina de Bioingeniería}
    \begin{rol}{Representante de Estudiantes}{Mar 2024 -- Oct 2025}
        \item Gestión de proyectos y eventos.
        \item Comunicación y difusión de eventos tecnológicos y científicos.
        %\item Fomento de la vinculación académica y profesional entre estudiantes de Bioingeniería a nivel nacional.
    \end{rol}
\end{experiencia}

\vfill
\begin{center}
    \footnotesize \color{black!50} Última actualización: \today \
\end{center}

\end{document}