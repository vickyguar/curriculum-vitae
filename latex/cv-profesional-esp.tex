\documentclass{article} % Fuente reducida a 9pt
\usepackage{config/config-esp} 

% Ajuste fino de márgenes de secciones para ganar espacio
\usepackage{titlesec}
\titlespacing{\section}{0pt}{1.2ex plus 0.2ex minus 0.2ex}{0.8ex plus 0.2ex}

\begin{document}
\pagestyle{empty}

% HEADER
% \noindent
% \begin{minipage}[t]{0.50\textwidth}
%     {\Huge \scshape{\color{primaryColor}Victoria Guarnieri}} \\[2pt]
%     {\large Ingeniera Biomédica}
% \end{minipage}
% \hfill
% \begin{minipage}[t]{0.48\textwidth}
%     \raggedleft
%     \footnotesize
%     \href{mailto:vickyguarnieri1@gmail.com}{vickyguarnieri1@gmail.com \faEnvelope} \\[1pt]
%     \href{tel:+5491130410475}{+54 9 11 3041-0475 \faPhone} \\[1pt]
%     \href{https://linkedin.com/in/victoria-guarnieri-}{victoria-guarnieri- \faLinkedin} \ $|$\
%     \href{https://github.com/vickyguar}{vickyguar \faGithub} \\[1pt]
%     Buenos Aires, Argentina \faMapMarker* \\[1pt]
%     DNI 43.660.111 \faIdCard
% \end{minipage}


% HEADER
\begin{center}
    {\huge \scshape{\color{primaryColor}Victoria Guarnieri}} \\ [5pt]
    {\large {Ingeniera Biomédica}} \\ [8pt]
    \small
    \href{mailto:vickyguarnieri1@gmail.com}{\faEnvelope \ vickyguarnieri1@gmail.com} \ $|$ \ 
    \href{tel:+5491130410475}{\faPhone \ +54 9 11 3041-0475} \ $|$ \ 
    \faMapMarker* \ Buenos Aires, Argentina \\ [3pt]
    \href{https://github.com/vickyguar}{\faGithub \ vickyguar} \ $|$ \ 
    \href{https://linkedin.com/in/victoria-guarnieri-}{\faLinkedin \ victoria-guarnieri-} \ $|$ \ 
    \faIdCard \ DNI 43.660.111
\end{center}

% FORMACIÓN ACADÉMICA
\section{Formación Académica}
\begin{formacion}{Ingeniería Biomédica}{Facultad de Ingeniería y Ciencias Exactas y Naturales}{Universidad Favaloro (FICEN-UF)}{Mar 2020 -- Dic 2025}
    \item \textbf{Promedio:} 8,35/10.
    \item \textbf{Proyecto Final:} Gestión de Datos en Investigación Oncológica: Desarrollo de Software Integral para Ensayos \textit{In Vivo}.
\end{formacion}
\tecnologias{C/C++, C\#, MATLAB, SQL.}\\
\habilidades{Trabajo en equipo, resolución de problemas, disciplina, perseverancia.}

% EXPERIENCIA LABORAL
\section{Experiencia Laboral}
\begin{experiencia}{Programa de Inteligencia Artificial en Salud, Hospital Italiano de Buenos Aires}
    \begin{rol}{Trainee en IA}{Sep 2024 -- Presente}
        \item Construcción y evaluación de un modelo de aprendizaje profundo para clasificación de imágenes.
        \item Diseño y ejecución de un estudio de nivel de acuerdo interobservador para definir etiquetas de entrenamiento.
        % \item Documentación técnica y seguimiento de experimentos.
        % \item Evaluación de productos externos.
    \end{rol}
\end{experiencia}
\tecnologias{\textbf{Python 3.13} (PyTorch, Pandas, Numpy, Scikit-learn, Matplotlib, Seaborn, Pyplot); \textbf{servicios de AWS} (AWS SageMaker, S3); \textbf{control de versiones} con Git, Github, Bitbucket; \textbf{seguimiento de experimentos} con Weights \& Biases (wandb); \textbf{R 4.3} (tidyverse, ggplot2, plotly, irrCAC).}\\
\habilidades{Análisis y visualización de datos, aprendizaje automático, aprendizaje profundo, procesamiento de imágenes, estadística aplicada.}

\vspace{6pt}
\begin{experiencia}{FICEN-UF -- Departamento de Matemática}
    % \begin{rol_corto}{Jefa de trabajos prácticos (JTP)}{Mar 2026 -- Presente}
    %     {JTP en la materia \textbf{Cálculo III} (Cálculo en variable compleja).}
    % \end{rol_corto}\\
    \begin{rol_corto}{Ayudante Alumno}{Ago 2021 -- Presente}
        {Ayudante en las materias \textbf{Cálculo I} (Cálculo en una variable), \textbf{Cálculo II} (Cálculo en varias variables) y \textbf{Cálculo III} (Cálculo en variable compleja).}
    \end{rol_corto}
\end{experiencia}

\vspace{6pt}
\begin{experiencia}{FICEN-UF -- Departamento de Informática}
    \begin{rol_corto}{Ayudante Alumno}{Mar 2025 -- Jul 2025}
        {Ayudante en la materia \textbf{Base de Datos}.}
    \end{rol_corto}
\end{experiencia}\\
\tecnologias{MySQL 8.0.}

\vspace{6pt}
\begin{experiencia}{Hospital Universitario Fundación Favaloro}
    \begin{rol}{Rotante en Ingeniería Biomédica}{May 2024 -- Sep 2024}
        \item Práctica profesional supervisada.
        \item Tareas de servicio técnico preventivo y correctivo. % Reparación a nivel placa y a nivel componentes; control de funcionamiento con testers biomédicos; control de seguridad eléctrica con analizador; compra de componentes a través del fondo fijo del sector.
        % \item Actualización de protocolos, conforme a la norma IRAM 15 y análisis de datos históricos de mantenimiento.
    \end{rol}
\end{experiencia}
%\tecnologias{Microsoft Excel, Neovero.}
%\habilidades{Ingeniería Clínica.}

% CURSOS
\section{Cursos}
\begin{small} % Cursos un poco más compactos
\curso{Deep Learning: Redes neuronales desde cero}{Centro de e-Learning UTN FRBA. Con evaluación.}{37}{Jul 2023}
\curso{Introducción al aprendizaje automático no supervisado}{FICEN-UF}{16}{May 2023}
\curso{Machine Learning con Python}{Centro de e-Learning UTN FRBA. Con evaluación.}{60}{May 2023}
\curso{Introducción a Python orientado a algoritmos y técnicas de programación competitiva}{IEEE UF, IEEE WIE y LambdaClass}{20}{Nov 2020}
\end{small}

% IDIOMAS
\section{Idiomas}
\textbf{Inglés:} Competencia profesional (lectura técnica y escritura científica).

% BECAS Y RECONOCIMIENTOS
\section{Becas y Reconocimientos}
\begin{itemize}[leftmargin=1.2em, label=\small\faAward, nosep]
    \item \normalsize\textbf{Beca de Investigación para Estudiantes (BIE)}, UF \fecha{Feb 2022 -- Feb 2025}
    \item \normalsize\textbf{Segunda escolta de bandera nacional}, FICEN-UF \fecha{Período lectivo 2024}
\end{itemize}

\vfill
\begin{center}
    \footnotesize \color{black!50} Última actualización: \today \
\end{center}

\end{document}