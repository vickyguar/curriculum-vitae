\documentclass{article} % Fuente reducida a 9pt
\usepackage{config-esp-resumido} 

% Ajuste fino de márgenes de secciones para ganar espacio
\usepackage{titlesec}
\titlespacing{\section}{0pt}{1.2ex plus 0.2ex minus 0.2ex}{0.8ex plus 0.2ex}

\begin{document}
\pagestyle{empty}

% HEADER REDISEÑADO
\noindent
\begin{minipage}[t]{0.60\textwidth}
    {\Huge \scshape{\color{primaryColor}Victoria Guarnieri}} \\[2pt]
    {\large Ingeniera Biomédica}
\end{minipage}
\hfill
\begin{minipage}[t]{0.38\textwidth}
    \scriptsize % Contacto un poco más pequeño
    \href{mailto:vickyguarnieri1@gmail.com}{\faEnvelope\ vickyguarnieri1@gmail.com} \\[1pt]
    \href{tel:+5491130410475}{\faPhone\ +54 9 11 3041-0475} \\[1pt]
    \href{https://github.com/vickyguar}{\faGithub\ vickyguar} \ $|$\
    \href{https://linkedin.com/in/victoria-guarnieri-}{\faLinkedin\ victoria-guarnieri-} \\[1pt]
    \faMapMarker*\ Buenos Aires, Argentina \\[1pt]
    \faIdCard\ DNI 43.660.111
\end{minipage}

\vspace{0.5em}

% FORMACIÓN ACADÉMICA
\section{Formación Académica}
\begin{formacion}{Ingeniería Biomédica}{Facultad de Ingeniería y Ciencias Exactas y Naturales}{Universidad Favaloro (FICEN-UF)}{Mar 2020 -- Dic 2025}
    \item \textbf{Promedio:} 8,35/10.
    \item \textbf{Proyecto Final:} Gestión de Datos en Investigación Oncológica: Desarrollo de Software Integral para Ensayos \textit{In Vivo}.
\end{formacion}
\tecnologias{C/C++, C\#, MATLAB, SQL.}
\habilidades{Trabajo en equipo, resolución de problemas, disciplina, perseverancia.}


% EXPERIENCIA LABORAL
\section{Experiencia Laboral}
\begin{experiencia}{Programa de Inteligencia Artificial en Salud, Hospital Italiano de Buenos Aires}{Trainee en IA}{Sep 2024 -- Presente}
    \item Construcción y evaluación de un modelo de aprendizaje profundo para clasificación, bajo supervisión.
    \item Diseño y ejecución de un estudio de grado de acuerdo interobservador para definir etiquetas de entrenamiento.
    % \item Documentación técnica y seguimiento de experimentos.
    % \item Evaluación de productos externos.
\end{experiencia}
\tecnologiasItems{
    \item \textbf{Python 3.13:} PyTorch, Pandas, Numpy, Scikit-learn, Matplotlib, Seaborn. \textbf{R 4.3:} tidyverse, ggplot2, plotly, irrCAC.
    \item \textbf{Cloud:} AWS SageMaker y AWS S3.
    \item \textbf{Control de versiones:} Git, Github y Bitbucket.
    \item \textbf{Seguimiento de experimentos:} Weights \& Biases (wandb).
}
\habilidades{Análisis y visualización de datos, aprendizaje automático, aprendizaje profundo, procesamiento de imágenes, estadística aplicada.}

\vspace{0.4em}

\begin{experiencia_corta}{FICEN-UF -- Departamento de Matemática}{Ayudante Alumno}{2021 -- Presente}{
    Ayudante en las materias \textbf{Cálculo I}, \textbf{Cálculo II} y \textbf{Cálculo III}.
}
\end{experiencia_corta}

\vspace{0.4em}

\begin{experiencia_corta}{FICEN-UF -- Departamento de Informática}{Ayudante Alumno}{Mar -- Jul 2025}{
    Ayudante en la materia \textbf{Base de Datos}.
}
\end{experiencia_corta}

\tecnologias{MySQL 8.0.}

\vspace{0.4em}

\begin{experiencia}{Hospital Universitario Fundación Favaloro}{Rotante en Ingeniería Biomédica}{May 2024 -- Sep 2024}
    \item Práctica profesional supervisada.
    \item Tareas de servicio técnico preventivo y correctivo. % Reparación a nivel placa y a nivel componentes; control de funcionamiento con testers biomédicos; control de seguridad eléctrica con analizador; compra de componentes a través del fondo fijo del sector.
    % \item Actualización de protocolos, conforme a la norma IRAM 15 y análisis de datos históricos de mantenimiento.
\end{experiencia}
%\tecnologias{Microsoft Excel, Neovero.}
%\habilidades{Ingeniería Clínica.}

% CURSOS
\section{Cursos}
\begin{small} % Cursos un poco más compactos
\curso{Deep Learning: Redes neuronales desde cero}{Centro de e-Learning UTN FRBA. Con evaluación.}{37}{Jul 2023}
\curso{Introducción al aprendizaje automático no supervisado}{FICEN-UF}{16}{May 2023}
\curso{Machine Learning con Python}{Centro de e-Learning UTN FRBA. Con evaluación.}{60}{May 2023}
\curso{Introducción a Python orientado a algoritmos y técnicas de programación competitiva}{IEEE UF, IEEE WIE y LambdaClass}{20}{Nov 2020}
\end{small}

% BECAS Y RECONOCIMIENTOS
\section{Becas y Reconocimientos}
\begin{itemize}[leftmargin=1.2em, label=\small\faAward, nosep]
    \item \normalsize\textbf{Beca de Investigación para Estudiantes (BIE)}, UF \fecha{Feb 2022 -- Feb 2025}
    \item \normalsize\textbf{Segunda escolta de bandera nacional}, FICEN-UF \fecha{Periodo lectivo 2024}
\end{itemize}

% IDIOMAS
\section{Idiomas}
\textbf{Inglés:} Competencia profesional (lectura técnica y escritura científica).

\vfill
\begin{center}
    \footnotesize \color{black!50} Última actualización: \today \
\end{center}

\end{document}